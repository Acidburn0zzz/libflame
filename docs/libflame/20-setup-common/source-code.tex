
%The \libflame source code is available in two forms: the previous milestone
%release and the latest nightly snapshots.
%{\bf Milestones and nightly snapshots.}
%We encourage users to download and use the latest nightly snapshot.
%These packages contain full copies of the \libflame source tree, including
%all the relevant build system scripts and makefiles.
%Though it may seem like the milestone releases would be more stable and the
%nightly snapshots more prone to bugs, this is not really the case.
%Milestone releases are fine immediately after they are released, but they
%quickly grow out-of-date.
%It is more likely that the bugs present in the previous milestone release
%have been identified and fixed in the latest nightly snapshot.
%Furthermore, we do not offer any ``updates'' to milestone releases in
%between major versions.
%It is for this reason that we strongly encourage users to use the latest
%nightly snapshot, especially if you encounter problems compiling the
%source code.
The \libflame source code is available via the web at {\tt github.com}:
\begin{Verbatim}[frame=none,framesep=2.5mm,xleftmargin=5mm,commandchars=\\\{\},fontsize=\normalsize]
http://www.github.com/flame/libflame/
\end{Verbatim}
We encourage users to download a copy of \libflame via the {\tt git clone}
command, rather than a gzipped-tarball.
That way, you can update your copy of libflame (via {\tt git pull} without
having to download an entirely new copy.

%If an error exists in an older release (milestone or nightly) and you suspect
%that the issue has not yet been addressed, please don't hesitate to contact
%the developers!
%You may email us anytime at \flameemailns.

%It is likely that a number of the bugs present in the previous milestone
%release have been identified and fixed in subsequent nightly snapshots.
%%It is for this reason that we strongly encourage users to use nightly
%%snapshots over milestones.%, especially if you encounter problems compiling the
%source code.
%Common problems such as errors while compiling code are likely to have
%already 

